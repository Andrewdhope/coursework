\documentclass[]{article}
\usepackage{lmodern}
\usepackage{amssymb,amsmath}
\usepackage{ifxetex,ifluatex}
\usepackage{fixltx2e} % provides \textsubscript
\ifnum 0\ifxetex 1\fi\ifluatex 1\fi=0 % if pdftex
  \usepackage[T1]{fontenc}
  \usepackage[utf8]{inputenc}
\else % if luatex or xelatex
  \ifxetex
    \usepackage{mathspec}
  \else
    \usepackage{fontspec}
  \fi
  \defaultfontfeatures{Ligatures=TeX,Scale=MatchLowercase}
\fi
% use upquote if available, for straight quotes in verbatim environments
\IfFileExists{upquote.sty}{\usepackage{upquote}}{}
% use microtype if available
\IfFileExists{microtype.sty}{%
\usepackage{microtype}
\UseMicrotypeSet[protrusion]{basicmath} % disable protrusion for tt fonts
}{}
\usepackage[margin=1in]{geometry}
\usepackage{hyperref}
\hypersetup{unicode=true,
            pdftitle={Inference Analysis},
            pdfauthor={Andrew Hope},
            pdfborder={0 0 0},
            breaklinks=true}
\urlstyle{same}  % don't use monospace font for urls
\usepackage{color}
\usepackage{fancyvrb}
\newcommand{\VerbBar}{|}
\newcommand{\VERB}{\Verb[commandchars=\\\{\}]}
\DefineVerbatimEnvironment{Highlighting}{Verbatim}{commandchars=\\\{\}}
% Add ',fontsize=\small' for more characters per line
\usepackage{framed}
\definecolor{shadecolor}{RGB}{248,248,248}
\newenvironment{Shaded}{\begin{snugshade}}{\end{snugshade}}
\newcommand{\KeywordTok}[1]{\textcolor[rgb]{0.13,0.29,0.53}{\textbf{{#1}}}}
\newcommand{\DataTypeTok}[1]{\textcolor[rgb]{0.13,0.29,0.53}{{#1}}}
\newcommand{\DecValTok}[1]{\textcolor[rgb]{0.00,0.00,0.81}{{#1}}}
\newcommand{\BaseNTok}[1]{\textcolor[rgb]{0.00,0.00,0.81}{{#1}}}
\newcommand{\FloatTok}[1]{\textcolor[rgb]{0.00,0.00,0.81}{{#1}}}
\newcommand{\ConstantTok}[1]{\textcolor[rgb]{0.00,0.00,0.00}{{#1}}}
\newcommand{\CharTok}[1]{\textcolor[rgb]{0.31,0.60,0.02}{{#1}}}
\newcommand{\SpecialCharTok}[1]{\textcolor[rgb]{0.00,0.00,0.00}{{#1}}}
\newcommand{\StringTok}[1]{\textcolor[rgb]{0.31,0.60,0.02}{{#1}}}
\newcommand{\VerbatimStringTok}[1]{\textcolor[rgb]{0.31,0.60,0.02}{{#1}}}
\newcommand{\SpecialStringTok}[1]{\textcolor[rgb]{0.31,0.60,0.02}{{#1}}}
\newcommand{\ImportTok}[1]{{#1}}
\newcommand{\CommentTok}[1]{\textcolor[rgb]{0.56,0.35,0.01}{\textit{{#1}}}}
\newcommand{\DocumentationTok}[1]{\textcolor[rgb]{0.56,0.35,0.01}{\textbf{\textit{{#1}}}}}
\newcommand{\AnnotationTok}[1]{\textcolor[rgb]{0.56,0.35,0.01}{\textbf{\textit{{#1}}}}}
\newcommand{\CommentVarTok}[1]{\textcolor[rgb]{0.56,0.35,0.01}{\textbf{\textit{{#1}}}}}
\newcommand{\OtherTok}[1]{\textcolor[rgb]{0.56,0.35,0.01}{{#1}}}
\newcommand{\FunctionTok}[1]{\textcolor[rgb]{0.00,0.00,0.00}{{#1}}}
\newcommand{\VariableTok}[1]{\textcolor[rgb]{0.00,0.00,0.00}{{#1}}}
\newcommand{\ControlFlowTok}[1]{\textcolor[rgb]{0.13,0.29,0.53}{\textbf{{#1}}}}
\newcommand{\OperatorTok}[1]{\textcolor[rgb]{0.81,0.36,0.00}{\textbf{{#1}}}}
\newcommand{\BuiltInTok}[1]{{#1}}
\newcommand{\ExtensionTok}[1]{{#1}}
\newcommand{\PreprocessorTok}[1]{\textcolor[rgb]{0.56,0.35,0.01}{\textit{{#1}}}}
\newcommand{\AttributeTok}[1]{\textcolor[rgb]{0.77,0.63,0.00}{{#1}}}
\newcommand{\RegionMarkerTok}[1]{{#1}}
\newcommand{\InformationTok}[1]{\textcolor[rgb]{0.56,0.35,0.01}{\textbf{\textit{{#1}}}}}
\newcommand{\WarningTok}[1]{\textcolor[rgb]{0.56,0.35,0.01}{\textbf{\textit{{#1}}}}}
\newcommand{\AlertTok}[1]{\textcolor[rgb]{0.94,0.16,0.16}{{#1}}}
\newcommand{\ErrorTok}[1]{\textcolor[rgb]{0.64,0.00,0.00}{\textbf{{#1}}}}
\newcommand{\NormalTok}[1]{{#1}}
\usepackage{graphicx,grffile}
\makeatletter
\def\maxwidth{\ifdim\Gin@nat@width>\linewidth\linewidth\else\Gin@nat@width\fi}
\def\maxheight{\ifdim\Gin@nat@height>\textheight\textheight\else\Gin@nat@height\fi}
\makeatother
% Scale images if necessary, so that they will not overflow the page
% margins by default, and it is still possible to overwrite the defaults
% using explicit options in \includegraphics[width, height, ...]{}
\setkeys{Gin}{width=\maxwidth,height=\maxheight,keepaspectratio}
\IfFileExists{parskip.sty}{%
\usepackage{parskip}
}{% else
\setlength{\parindent}{0pt}
\setlength{\parskip}{6pt plus 2pt minus 1pt}
}
\setlength{\emergencystretch}{3em}  % prevent overfull lines
\providecommand{\tightlist}{%
  \setlength{\itemsep}{0pt}\setlength{\parskip}{0pt}}
\setcounter{secnumdepth}{0}
% Redefines (sub)paragraphs to behave more like sections
\ifx\paragraph\undefined\else
\let\oldparagraph\paragraph
\renewcommand{\paragraph}[1]{\oldparagraph{#1}\mbox{}}
\fi
\ifx\subparagraph\undefined\else
\let\oldsubparagraph\subparagraph
\renewcommand{\subparagraph}[1]{\oldsubparagraph{#1}\mbox{}}
\fi

%%% Use protect on footnotes to avoid problems with footnotes in titles
\let\rmarkdownfootnote\footnote%
\def\footnote{\protect\rmarkdownfootnote}

%%% Change title format to be more compact
\usepackage{titling}

% Create subtitle command for use in maketitle
\newcommand{\subtitle}[1]{
  \posttitle{
    \begin{center}\large#1\end{center}
    }
}

\setlength{\droptitle}{-2em}
  \title{Inference Analysis}
  \pretitle{\vspace{\droptitle}\centering\huge}
  \posttitle{\par}
  \author{Andrew Hope}
  \preauthor{\centering\large\emph}
  \postauthor{\par}
  \predate{\centering\large\emph}
  \postdate{\par}
  \date{April 6, 2018}


\begin{document}
\maketitle

\section{Inference Analysis With an Exponential
Distribution}\label{inference-analysis-with-an-exponential-distribution}

\subsection{Overview}\label{overview}

This analysis will demonstrate inferential statistics techniques to
evaluate sampling mean and variance for a large sample with an
underlying exponential distribution. The analysis will begin by creating
a simulated dataset, and will then calculate and compare sampling
statistics. A final section provides plots to better convey the
conclusions.

\begin{Shaded}
\begin{Highlighting}[]
\KeywordTok{library}\NormalTok{(ggplot2)}
\end{Highlighting}
\end{Shaded}

\begin{verbatim}
## Warning: package 'ggplot2' was built under R version 3.4.3
\end{verbatim}

\subsection{Simulations}\label{simulations}

Create a set of 1000 random samples from an exponential distribution
with lambda equal to 0.2. Each sample has a size n = 40. Begin by
generating all of the needed data points. In this analysis we round the
number to 4 decimal places for the sake of keeping the data simpler and
cleaner.

\begin{Shaded}
\begin{Highlighting}[]
\NormalTok{r <-}\StringTok{ }\KeywordTok{round}\NormalTok{(}\KeywordTok{rexp}\NormalTok{(}\DecValTok{40000}\NormalTok{, }\DataTypeTok{rate =} \FloatTok{0.2}\NormalTok{), }\DecValTok{4}\NormalTok{)}
\end{Highlighting}
\end{Shaded}

Move the values into a matrix to create 1000 samples of size n = 40.

\begin{Shaded}
\begin{Highlighting}[]
\NormalTok{rmat <-}\StringTok{ }\KeywordTok{matrix}\NormalTok{(r, }\DecValTok{1000}\NormalTok{, }\DecValTok{40}\NormalTok{)}
\KeywordTok{head}\NormalTok{(rmat, }\DecValTok{3}\NormalTok{)}
\end{Highlighting}
\end{Shaded}

\begin{verbatim}
##        [,1]   [,2]   [,3]    [,4]   [,5]    [,6]    [,7]    [,8]    [,9]
## [1,] 2.1672 1.8437 0.1866 10.4820 4.5321  3.8135  0.0241  1.0514  8.6790
## [2,] 0.6383 3.0988 0.1111  1.9064 3.5363  3.6789 10.1816  4.7340 21.4870
## [3,] 8.4726 1.0741 4.2327  4.2313 2.3475 23.5309  1.8820 12.2457  1.3906
##       [,10]  [,11]  [,12]   [,13]   [,14]   [,15]  [,16]  [,17]  [,18]
## [1,] 3.1105 1.7616 0.9355  7.1099  0.9891 11.5178 3.7403 4.2998 2.5795
## [2,] 1.0742 2.9253 3.1100  3.3301 14.4693  0.2964 3.3991 4.1641 3.9781
## [3,] 2.7505 8.0905 0.0827 16.5348  3.3730  1.0950 1.1466 4.4325 2.4740
##       [,19]  [,20]   [,21]  [,22]  [,23]  [,24]  [,25]  [,26]  [,27]
## [1,] 0.6485 0.5684  2.2255 3.3269 1.3138 0.3945 0.6135 5.9325 3.5954
## [2,] 0.9981 0.5735  3.7088 0.5246 7.2426 1.7546 5.1498 6.2606 7.6422
## [3,] 0.9567 4.7932 14.6355 4.1966 0.0504 0.1900 0.2111 2.2373 1.5750
##        [,28]  [,29]  [,30]  [,31]   [,32]  [,33]  [,34]  [,35]  [,36]
## [1,]  7.7665 0.5858 3.8820 2.3217  6.4724 3.4360 2.2588 0.5789 7.6784
## [2,] 11.3952 1.7090 0.1137 4.2756 16.0404 6.0566 2.8335 0.3730 8.0536
## [3,]  4.5294 4.2936 1.4089 4.8257  9.3983 0.8138 1.5171 4.5116 2.1887
##       [,37]  [,38]   [,39]  [,40]
## [1,] 3.3978 3.0393  2.7735 1.3218
## [2,] 5.7616 1.7530 10.0791 3.0520
## [3,] 0.1446 1.8129  1.8344 0.1202
\end{verbatim}

\subsection{Sample Mean vs.~Theoretical
Mean}\label{sample-mean-vs.theoretical-mean}

Find the mean for each of the 1000 rows, and take the mean of means to
find the sampling mean.

\begin{Shaded}
\begin{Highlighting}[]
\NormalTok{sampling <-}\StringTok{ }\KeywordTok{apply}\NormalTok{(rmat, }\DecValTok{1}\NormalTok{, mean)}
\NormalTok{m <-}\StringTok{ }\KeywordTok{mean}\NormalTok{(sampling)}
\NormalTok{m}
\end{Highlighting}
\end{Shaded}

\begin{verbatim}
## [1] 4.958667
\end{verbatim}

The Central Limit Theorm allows us to conclude that this sampling mean
is approximately equal to the theoretical mean.

The theoretical mean of an exponential distribution is equal to
1/lambda.

\begin{Shaded}
\begin{Highlighting}[]
\DecValTok{1}\NormalTok{/}\FloatTok{0.2}
\end{Highlighting}
\end{Shaded}

\begin{verbatim}
## [1] 5
\end{verbatim}

Indeed, our sampling mean is very close to the theoretical mean.

\subsection{Sample Variance vs.~Thoretical
Variance}\label{sample-variance-vs.thoretical-variance}

Calculate the population variance.

\begin{Shaded}
\begin{Highlighting}[]
\NormalTok{r.var <-}\StringTok{ }\KeywordTok{var}\NormalTok{(r)}
\NormalTok{r.var}
\end{Highlighting}
\end{Shaded}

\begin{verbatim}
## [1] 24.2146
\end{verbatim}

This should be close to the theoretical variance of an exponential
distribution, which is equal to 1/lambda\^{}2.

\begin{Shaded}
\begin{Highlighting}[]
\DecValTok{1}\NormalTok{/}\FloatTok{0.2}\NormalTok{^}\DecValTok{2}
\end{Highlighting}
\end{Shaded}

\begin{verbatim}
## [1] 25
\end{verbatim}

Calculate the variance of the 1000 sample means to find the sampling
variance.

\begin{Shaded}
\begin{Highlighting}[]
\NormalTok{v <-}\StringTok{ }\KeywordTok{var}\NormalTok{(sampling)}
\NormalTok{v}
\end{Highlighting}
\end{Shaded}

\begin{verbatim}
## [1] 0.6253581
\end{verbatim}

This variance of the sample means should be approximately equal to its
theoretical equivalent, (1/lambda\^{}2)/n

\begin{Shaded}
\begin{Highlighting}[]
\NormalTok{(}\DecValTok{1}\NormalTok{/}\FloatTok{0.2}\NormalTok{^}\DecValTok{2}\NormalTok{)/}\DecValTok{40}
\end{Highlighting}
\end{Shaded}

\begin{verbatim}
## [1] 0.625
\end{verbatim}

\subsection{Distribution}\label{distribution}

There is a significant difference between the distribution of an
exponential random variable, and the distribution of means of a set of
samples of exponential random variables.

First, look at a distribution of a single exponential random variable.

\begin{Shaded}
\begin{Highlighting}[]
\KeywordTok{qplot}\NormalTok{(r, }\DataTypeTok{bins =} \DecValTok{30}\NormalTok{)}
\end{Highlighting}
\end{Shaded}

\includegraphics{Simulation_Exercise_files/figure-latex/unnamed-chunk-6-1.pdf}

Now, look at the distribution of the 1000 mean values from the simulated
data set.

\begin{Shaded}
\begin{Highlighting}[]
\KeywordTok{qplot}\NormalTok{(sampling, }\DataTypeTok{bins =} \DecValTok{30}\NormalTok{)}
\end{Highlighting}
\end{Shaded}

\includegraphics{Simulation_Exercise_files/figure-latex/unnamed-chunk-7-1.pdf}

This demonstrates the Central Limit Threorem. The distribution of a
sample's statistic, across a set of samples, will resemble a normal
distribution.


\end{document}
