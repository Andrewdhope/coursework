\documentclass[]{article}
\usepackage{lmodern}
\usepackage{amssymb,amsmath}
\usepackage{ifxetex,ifluatex}
\usepackage{fixltx2e} % provides \textsubscript
\ifnum 0\ifxetex 1\fi\ifluatex 1\fi=0 % if pdftex
  \usepackage[T1]{fontenc}
  \usepackage[utf8]{inputenc}
\else % if luatex or xelatex
  \ifxetex
    \usepackage{mathspec}
  \else
    \usepackage{fontspec}
  \fi
  \defaultfontfeatures{Ligatures=TeX,Scale=MatchLowercase}
\fi
% use upquote if available, for straight quotes in verbatim environments
\IfFileExists{upquote.sty}{\usepackage{upquote}}{}
% use microtype if available
\IfFileExists{microtype.sty}{%
\usepackage{microtype}
\UseMicrotypeSet[protrusion]{basicmath} % disable protrusion for tt fonts
}{}
\usepackage[margin=1in]{geometry}
\usepackage{hyperref}
\hypersetup{unicode=true,
            pdftitle={Inference Analysis},
            pdfauthor={Andrew Hope},
            pdfborder={0 0 0},
            breaklinks=true}
\urlstyle{same}  % don't use monospace font for urls
\usepackage{color}
\usepackage{fancyvrb}
\newcommand{\VerbBar}{|}
\newcommand{\VERB}{\Verb[commandchars=\\\{\}]}
\DefineVerbatimEnvironment{Highlighting}{Verbatim}{commandchars=\\\{\}}
% Add ',fontsize=\small' for more characters per line
\usepackage{framed}
\definecolor{shadecolor}{RGB}{248,248,248}
\newenvironment{Shaded}{\begin{snugshade}}{\end{snugshade}}
\newcommand{\KeywordTok}[1]{\textcolor[rgb]{0.13,0.29,0.53}{\textbf{{#1}}}}
\newcommand{\DataTypeTok}[1]{\textcolor[rgb]{0.13,0.29,0.53}{{#1}}}
\newcommand{\DecValTok}[1]{\textcolor[rgb]{0.00,0.00,0.81}{{#1}}}
\newcommand{\BaseNTok}[1]{\textcolor[rgb]{0.00,0.00,0.81}{{#1}}}
\newcommand{\FloatTok}[1]{\textcolor[rgb]{0.00,0.00,0.81}{{#1}}}
\newcommand{\ConstantTok}[1]{\textcolor[rgb]{0.00,0.00,0.00}{{#1}}}
\newcommand{\CharTok}[1]{\textcolor[rgb]{0.31,0.60,0.02}{{#1}}}
\newcommand{\SpecialCharTok}[1]{\textcolor[rgb]{0.00,0.00,0.00}{{#1}}}
\newcommand{\StringTok}[1]{\textcolor[rgb]{0.31,0.60,0.02}{{#1}}}
\newcommand{\VerbatimStringTok}[1]{\textcolor[rgb]{0.31,0.60,0.02}{{#1}}}
\newcommand{\SpecialStringTok}[1]{\textcolor[rgb]{0.31,0.60,0.02}{{#1}}}
\newcommand{\ImportTok}[1]{{#1}}
\newcommand{\CommentTok}[1]{\textcolor[rgb]{0.56,0.35,0.01}{\textit{{#1}}}}
\newcommand{\DocumentationTok}[1]{\textcolor[rgb]{0.56,0.35,0.01}{\textbf{\textit{{#1}}}}}
\newcommand{\AnnotationTok}[1]{\textcolor[rgb]{0.56,0.35,0.01}{\textbf{\textit{{#1}}}}}
\newcommand{\CommentVarTok}[1]{\textcolor[rgb]{0.56,0.35,0.01}{\textbf{\textit{{#1}}}}}
\newcommand{\OtherTok}[1]{\textcolor[rgb]{0.56,0.35,0.01}{{#1}}}
\newcommand{\FunctionTok}[1]{\textcolor[rgb]{0.00,0.00,0.00}{{#1}}}
\newcommand{\VariableTok}[1]{\textcolor[rgb]{0.00,0.00,0.00}{{#1}}}
\newcommand{\ControlFlowTok}[1]{\textcolor[rgb]{0.13,0.29,0.53}{\textbf{{#1}}}}
\newcommand{\OperatorTok}[1]{\textcolor[rgb]{0.81,0.36,0.00}{\textbf{{#1}}}}
\newcommand{\BuiltInTok}[1]{{#1}}
\newcommand{\ExtensionTok}[1]{{#1}}
\newcommand{\PreprocessorTok}[1]{\textcolor[rgb]{0.56,0.35,0.01}{\textit{{#1}}}}
\newcommand{\AttributeTok}[1]{\textcolor[rgb]{0.77,0.63,0.00}{{#1}}}
\newcommand{\RegionMarkerTok}[1]{{#1}}
\newcommand{\InformationTok}[1]{\textcolor[rgb]{0.56,0.35,0.01}{\textbf{\textit{{#1}}}}}
\newcommand{\WarningTok}[1]{\textcolor[rgb]{0.56,0.35,0.01}{\textbf{\textit{{#1}}}}}
\newcommand{\AlertTok}[1]{\textcolor[rgb]{0.94,0.16,0.16}{{#1}}}
\newcommand{\ErrorTok}[1]{\textcolor[rgb]{0.64,0.00,0.00}{\textbf{{#1}}}}
\newcommand{\NormalTok}[1]{{#1}}
\usepackage{graphicx,grffile}
\makeatletter
\def\maxwidth{\ifdim\Gin@nat@width>\linewidth\linewidth\else\Gin@nat@width\fi}
\def\maxheight{\ifdim\Gin@nat@height>\textheight\textheight\else\Gin@nat@height\fi}
\makeatother
% Scale images if necessary, so that they will not overflow the page
% margins by default, and it is still possible to overwrite the defaults
% using explicit options in \includegraphics[width, height, ...]{}
\setkeys{Gin}{width=\maxwidth,height=\maxheight,keepaspectratio}
\IfFileExists{parskip.sty}{%
\usepackage{parskip}
}{% else
\setlength{\parindent}{0pt}
\setlength{\parskip}{6pt plus 2pt minus 1pt}
}
\setlength{\emergencystretch}{3em}  % prevent overfull lines
\providecommand{\tightlist}{%
  \setlength{\itemsep}{0pt}\setlength{\parskip}{0pt}}
\setcounter{secnumdepth}{0}
% Redefines (sub)paragraphs to behave more like sections
\ifx\paragraph\undefined\else
\let\oldparagraph\paragraph
\renewcommand{\paragraph}[1]{\oldparagraph{#1}\mbox{}}
\fi
\ifx\subparagraph\undefined\else
\let\oldsubparagraph\subparagraph
\renewcommand{\subparagraph}[1]{\oldsubparagraph{#1}\mbox{}}
\fi

%%% Use protect on footnotes to avoid problems with footnotes in titles
\let\rmarkdownfootnote\footnote%
\def\footnote{\protect\rmarkdownfootnote}

%%% Change title format to be more compact
\usepackage{titling}

% Create subtitle command for use in maketitle
\newcommand{\subtitle}[1]{
  \posttitle{
    \begin{center}\large#1\end{center}
    }
}

\setlength{\droptitle}{-2em}
  \title{Inference Analysis}
  \pretitle{\vspace{\droptitle}\centering\huge}
  \posttitle{\par}
  \author{Andrew Hope}
  \preauthor{\centering\large\emph}
  \postauthor{\par}
  \predate{\centering\large\emph}
  \postdate{\par}
  \date{April 6, 2018}


\begin{document}
\maketitle

\section{Inference Analysis Comparing Partitioned
Data}\label{inference-analysis-comparing-partitioned-data}

\subsection{Overview}\label{overview}

This analysis aims to determine if there is a significant difference in
mean tooth length based on two different variables. The analysis will
first describe the dataset, and will then use an independent two-sample
t-test to look for significant differences between subset means. The
anlysis will use an alpha value of 0.5, and it assumes that the samples
were taken randomly from a population with an approximately normal
distribution.

\subsection{Load Data}\label{load-data}

Load in the ToothGrowth dataset from the R dataset library.

\begin{Shaded}
\begin{Highlighting}[]
    \KeywordTok{library}\NormalTok{(datasets)}
    \NormalTok{tg <-}\StringTok{ }\NormalTok{ToothGrowth}
\end{Highlighting}
\end{Shaded}

\subsection{Exploratory Analysis and
Transformations}\label{exploratory-analysis-and-transformations}

Use some basic exploratory techniques to understand this small dataset.

\begin{Shaded}
\begin{Highlighting}[]
    \KeywordTok{dim}\NormalTok{(tg)}
\end{Highlighting}
\end{Shaded}

\begin{verbatim}
## [1] 60  3
\end{verbatim}

\begin{Shaded}
\begin{Highlighting}[]
    \KeywordTok{summary}\NormalTok{(tg)}
\end{Highlighting}
\end{Shaded}

\begin{verbatim}
##       len        supp         dose      
##  Min.   : 4.20   OJ:30   Min.   :0.500  
##  1st Qu.:13.07   VC:30   1st Qu.:0.500  
##  Median :19.25           Median :1.000  
##  Mean   :18.81           Mean   :1.167  
##  3rd Qu.:25.27           3rd Qu.:2.000  
##  Max.   :33.90           Max.   :2.000
\end{verbatim}

The quartiles for the dose column are a bit unusual in the summary. They
are multiples of 0.5. Explore this further.

\begin{Shaded}
\begin{Highlighting}[]
    \KeywordTok{unique}\NormalTok{(tg$dose)}
\end{Highlighting}
\end{Shaded}

\begin{verbatim}
## [1] 0.5 1.0 2.0
\end{verbatim}

This variable should be a factor rather than numeric. Perform this
transformation.

\begin{Shaded}
\begin{Highlighting}[]
    \NormalTok{tg$dose <-}\StringTok{ }\KeywordTok{as.factor}\NormalTok{(tg$dose)}
    \KeywordTok{summary}\NormalTok{(tg)}
\end{Highlighting}
\end{Shaded}

\begin{verbatim}
##       len        supp     dose   
##  Min.   : 4.20   OJ:30   0.5:20  
##  1st Qu.:13.07   VC:30   1  :20  
##  Median :19.25           2  :20  
##  Mean   :18.81                   
##  3rd Qu.:25.27                   
##  Max.   :33.90
\end{verbatim}

\subsection{Calculate Statistics}\label{calculate-statistics}

Split the data by supp and dose.

\begin{Shaded}
\begin{Highlighting}[]
    \NormalTok{supp <-}\StringTok{ }\KeywordTok{split}\NormalTok{(tg$len, tg$supp)}
    \NormalTok{dose <-}\StringTok{ }\KeywordTok{split}\NormalTok{(tg$len, tg$dose)}
\end{Highlighting}
\end{Shaded}

Calculate the mean, variance, and standard deviation for each value of
supp.

\begin{Shaded}
\begin{Highlighting}[]
    \NormalTok{ms <-}\StringTok{ }\KeywordTok{tapply}\NormalTok{(tg$len, tg$supp, mean)}
    \NormalTok{vs <-}\StringTok{ }\KeywordTok{tapply}\NormalTok{(tg$len, tg$supp, var)}
    \NormalTok{ms}
\end{Highlighting}
\end{Shaded}

\begin{verbatim}
##       OJ       VC 
## 20.66333 16.96333
\end{verbatim}

\begin{Shaded}
\begin{Highlighting}[]
    \NormalTok{vs}
\end{Highlighting}
\end{Shaded}

\begin{verbatim}
##       OJ       VC 
## 43.63344 68.32723
\end{verbatim}

\begin{Shaded}
\begin{Highlighting}[]
    \KeywordTok{sqrt}\NormalTok{(vs)}
\end{Highlighting}
\end{Shaded}

\begin{verbatim}
##       OJ       VC 
## 6.605561 8.266029
\end{verbatim}

Do the same with dose.

\begin{Shaded}
\begin{Highlighting}[]
    \NormalTok{md <-}\StringTok{ }\KeywordTok{tapply}\NormalTok{(tg$len, tg$dose, mean)}
    \NormalTok{vd <-}\StringTok{ }\KeywordTok{tapply}\NormalTok{(tg$len, tg$dose, var)}
    \NormalTok{md}
\end{Highlighting}
\end{Shaded}

\begin{verbatim}
##    0.5      1      2 
## 10.605 19.735 26.100
\end{verbatim}

\begin{Shaded}
\begin{Highlighting}[]
    \NormalTok{vd}
\end{Highlighting}
\end{Shaded}

\begin{verbatim}
##      0.5        1        2 
## 20.24787 19.49608 14.24421
\end{verbatim}

\begin{Shaded}
\begin{Highlighting}[]
    \KeywordTok{sqrt}\NormalTok{(vd)}
\end{Highlighting}
\end{Shaded}

\begin{verbatim}
##      0.5        1        2 
## 4.499763 4.415436 3.774150
\end{verbatim}

\subsection{Compare Mean of
Partitions}\label{compare-mean-of-partitions}

Compare the means (grouped by supp) using a 95\% confidence interval and
an independent two-tailed t-test.

\begin{Shaded}
\begin{Highlighting}[]
\NormalTok{supp.t <-}\StringTok{ }\KeywordTok{t.test}\NormalTok{(supp$OJ, supp$VC, }\DataTypeTok{conf.level =} \FloatTok{0.95}\NormalTok{)}
\NormalTok{supp.t}
\end{Highlighting}
\end{Shaded}

\begin{verbatim}
## 
##  Welch Two Sample t-test
## 
## data:  supp$OJ and supp$VC
## t = 1.9153, df = 55.309, p-value = 0.06063
## alternative hypothesis: true difference in means is not equal to 0
## 95 percent confidence interval:
##  -0.1710156  7.5710156
## sample estimates:
## mean of x mean of y 
##  20.66333  16.96333
\end{verbatim}

For dose, execute an independent two-sample t-test between each factor
level. This will result in three comparisons: 0.5 and 1, 0.5 and 2, and
1 and 2.

First comparison -- factor level 0.5 to 1.

\begin{Shaded}
\begin{Highlighting}[]
\NormalTok{dose.t12 <-}\StringTok{ }\KeywordTok{t.test}\NormalTok{(dose$}\StringTok{`}\DataTypeTok{0.5}\StringTok{`}\NormalTok{, dose$}\StringTok{`}\DataTypeTok{1}\StringTok{`}\NormalTok{, }\DataTypeTok{conf.level =} \FloatTok{0.95}\NormalTok{)}
\NormalTok{dose.t12}
\end{Highlighting}
\end{Shaded}

\begin{verbatim}
## 
##  Welch Two Sample t-test
## 
## data:  dose$`0.5` and dose$`1`
## t = -6.4766, df = 37.986, p-value = 1.268e-07
## alternative hypothesis: true difference in means is not equal to 0
## 95 percent confidence interval:
##  -11.983781  -6.276219
## sample estimates:
## mean of x mean of y 
##    10.605    19.735
\end{verbatim}

Second comparison -- factor level 0.5 to 2.

\begin{Shaded}
\begin{Highlighting}[]
\NormalTok{dose.t13 <-}\StringTok{ }\KeywordTok{t.test}\NormalTok{(dose$}\StringTok{`}\DataTypeTok{0.5}\StringTok{`}\NormalTok{, dose$}\StringTok{`}\DataTypeTok{2}\StringTok{`}\NormalTok{, }\DataTypeTok{conf.level =} \FloatTok{0.95}\NormalTok{)}
\NormalTok{dose.t13}
\end{Highlighting}
\end{Shaded}

\begin{verbatim}
## 
##  Welch Two Sample t-test
## 
## data:  dose$`0.5` and dose$`2`
## t = -11.799, df = 36.883, p-value = 4.398e-14
## alternative hypothesis: true difference in means is not equal to 0
## 95 percent confidence interval:
##  -18.15617 -12.83383
## sample estimates:
## mean of x mean of y 
##    10.605    26.100
\end{verbatim}

Third comparison -- factor level 1 to 2.

\begin{Shaded}
\begin{Highlighting}[]
\NormalTok{dose.t23 <-}\StringTok{ }\KeywordTok{t.test}\NormalTok{(dose$}\StringTok{`}\DataTypeTok{1}\StringTok{`}\NormalTok{, dose$}\StringTok{`}\DataTypeTok{2}\StringTok{`}\NormalTok{, }\DataTypeTok{conf.level =} \FloatTok{0.95}\NormalTok{)}
\NormalTok{dose.t23}
\end{Highlighting}
\end{Shaded}

\begin{verbatim}
## 
##  Welch Two Sample t-test
## 
## data:  dose$`1` and dose$`2`
## t = -4.9005, df = 37.101, p-value = 1.906e-05
## alternative hypothesis: true difference in means is not equal to 0
## 95 percent confidence interval:
##  -8.996481 -3.733519
## sample estimates:
## mean of x mean of y 
##    19.735    26.100
\end{verbatim}

\subsection{Conclusions}\label{conclusions}

For the supp variable, using a 95\% confidence interval does not allow
us to reject a null hypothesis. The two groups do not have significantly
different means. This is indicated by the confidence interval containing
zero, and also the p-value being greater than our accepted alpha of 0.5.

\begin{Shaded}
\begin{Highlighting}[]
    \NormalTok{supp.t$conf.int}
\end{Highlighting}
\end{Shaded}

\begin{verbatim}
## [1] -0.1710156  7.5710156
## attr(,"conf.level")
## [1] 0.95
\end{verbatim}

\begin{Shaded}
\begin{Highlighting}[]
    \NormalTok{supp.t$p.value}
\end{Highlighting}
\end{Shaded}

\begin{verbatim}
## [1] 0.06063451
\end{verbatim}

For the dose variable, each comparison gave a confidence interval that
did not include zero. This indicates that the groups have significantly
different means. For all three comparisons, the p-value is less than
0.5, allowing us to comfortably reject the null hypothesis.

\begin{Shaded}
\begin{Highlighting}[]
    \NormalTok{dose.t12$p.value}
\end{Highlighting}
\end{Shaded}

\begin{verbatim}
## [1] 1.268301e-07
\end{verbatim}

\begin{Shaded}
\begin{Highlighting}[]
    \NormalTok{dose.t13$p.value}
\end{Highlighting}
\end{Shaded}

\begin{verbatim}
## [1] 4.397525e-14
\end{verbatim}

\begin{Shaded}
\begin{Highlighting}[]
    \NormalTok{dose.t23$p.value}
\end{Highlighting}
\end{Shaded}

\begin{verbatim}
## [1] 1.90643e-05
\end{verbatim}


\end{document}
